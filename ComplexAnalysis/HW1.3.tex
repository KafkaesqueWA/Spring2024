\documentclass{article}
\usepackage[margin=2cm]{geometry}
\usepackage[utf8]{inputenc}
\usepackage{amsfonts}
\usepackage{amsmath}
\usepackage{multicol}
\usepackage{amsthm}
\usepackage{amssymb}
\usepackage{graphicx}
\newtheorem{theorem}{Theorum}[section]
\newtheorem{corollary}{Corollary}[theorem]
\newtheorem{lemma}[theorem]{Lemma}
\title{Homework Set 1.3}
\author{Warren Atkison}
\date{\today}
\begin{document}

\maketitle

\section*{Exercise 11}
Using the complex product $(1 + i)(5 - i)^4$, derive
\[
	\pi/4 = 4\tan^{-1}(1/5) - \tan^{-1}(1/239)
\]
First we shall solve for the product.
\begin{align*}
	(1 + i)(5 - i)^4 &= (1 + i)(24 - 10i)^2 \\
			 &= (1 + i)(476 - 480i) \\
			 &= 956 - 4i
\end{align*}
Since both $1 + i$, $5 - i$, and $956 - 4i$ are in either quadraints I or IV, we can use $\arg(x + iy) = \tan^{-1}(y/x)$.
\begin{align*}	
	\arg((1 + i)(5-i)^4) &= \arg(1 + i) + 4\arg(5 - i) = \arg(956 - 4i) \\
			     &= \tan^{-1}(1) + 4\tan^{1}(-1/5) = \tan^{-1}(-4/956) \\
			     &= \pi/4 - 4\tan(1/5) = -\tan^{-1}(1/239)
\end{align*}
Finally we can re-arange our eqaution to get
\[
	\pi/4 = 4\tan^{-1}(1/5) - \tan^{-1}(1/239)	
\]
\newpage
\section*{Exercise 7d}
Find the argument of the following complex number and write in polar form.
\[
	z = -2\sqrt{3}-2i 
\]
\begin{align*}
	r &= |z| = \sqrt{12 + 4} = 4 \\
	\cos(\theta) &= -2\sqrt{3} /4 = -\sqrt{3}/2 & \sin(\theta) &= -2/4 = -1/2 \\
	\theta &= 7\pi/6 \\
	z &= 4(\cos(7\pi/3) + i\sin(7\pi/3))
\end{align*}

\end{document}
