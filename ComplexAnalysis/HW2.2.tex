\documentclass[12pt]{article}
\linespread{1.2}
\usepackage[margin=2cm]{geometry}
\usepackage[utf8]{inputenc}
\usepackage{amsfonts}
\usepackage{amsmath}
\usepackage{multicol}
\usepackage{amsthm}
\usepackage{amssymb,scrextend}
\usepackage{graphicx,tikz,pgfplots}
\pgfplotsset{compat=1.18}
\newtheorem{dfn}{Definition}
\renewcommand{\qed}{\hfill$\blacksquare$}
\let\newproof\proof
\renewenvironment{proof}{\vspace{1em}\begin{addmargin}[2em]{0em}\begin{newproof}}{\end{newproof}\end{addmargin}\qed}
\newenvironment{theorem}[2][Theorem]{\begin{trivlist}
\item[\hskip \labelsep {\bfseries #1} \hskip \labelsep {\bfseries #2.}]}{\end{trivlist}}
\newenvironment{example}[2][Example]{\begin{trivlist}
\item[\hskip \labelsep {\bfseries #1} \hskip \labelsep {\bfseries #2.}]}{\end{trivlist}}
\newenvironment{lemma}[2][Lemma]{\begin{trivlist}
\item[\hskip \labelsep {\bfseries #1} \hskip \labelsep {\bfseries #2.}]}{\end{trivlist}}
\newenvironment{exercise}[2][Exercise]{\begin{trivlist}
\item[\hskip \labelsep {\bfseries #1} \hskip \labelsep {\bfseries #2.}]}{\end{trivlist}}
\newenvironment{problem}[2][Problem]{\begin{trivlist}
\item[\hskip \labelsep {\bfseries #1} \hskip \labelsep {\bfseries #2.}]}{\end{trivlist}}
\newenvironment{corollary}[2][Corollary]{\begin{trivlist}
\item[\hskip \labelsep {\bfseries #1} \hskip \labelsep {\bfseries #2.}]}{\end{trivlist}}
\usepackage{fancyhdr,enumitem,changepage,url}
\pagestyle{fancy}
\author{Warren Atkison}
\date{\today}
\setlength{\headheight}{15pt}
\begin{document}
\fancyhf{}
\fancyhead[L]{Warren Atkison}
\fancyhead[C]{Homework Set 2.2}
\fancyhead[R]{\today}
\fancyfoot[R]{\thepage}

\begin{exercise}{2}
	Sketch the first five terms of the sequence $(2i)^n, n = 1,~2,~3,\ldots$, and describe the divergence of this sequence
\end{exercise}
\begin{center}
	\begin{tikzpicture}
		\begin{axis}[xmin = -35, xmax = 35, ymin = -35, ymax = 35, axis x line = middle, axis y line = middle]
			\addplot[mark=*] coordinates {(0,2)};
			\addplot[mark=*] coordinates {(-4,0)};
			\addplot[mark=*] coordinates {(0,-8)};
			\addplot[mark=*] coordinates {(16,0)};
			\addplot[mark=*] coordinates {(0,32)};
		\end{axis}	
	\end{tikzpicture}
\end{center}
This sequence doubles in magnitude and rotates $90^{\circ}$ on each iteration, approaching $\pm \infty$ on both the real and imaginary axis.
\begin{exercise}{8}
	Use the Definition 2 to prove that $\lim_{z \to 1 + i} (6z - 4) = 2 + 6i$
\end{exercise}	
\begin{proof} Let $\delta = \varepsilon/6$. We shall show that for all $\varepsilon > 0$
	\[
		|(6z - 4) - (2 + 6i)| < \varepsilon
	\]
	whenever $0 < |z - (1 + i)| < \delta$.
	\begin{align*}
		|(6z - 4) - (2 + 6i)| &= |6z - 6 - 6i| \\ &= 6|z - (1 + i)| \\
						      &< 6\delta \\
				\implies |z - (1 + i)| &< \delta
	\end{align*}
\end{proof}
\begin{exercise}{11c}
	Find the following limit.
	\[
		\lim_{z \to 3i} \frac{z^2 + 9}{z - 3i}
	\]
	The function $\dfrac{z^2 + 9}{z - 3i}$ is not continuous at $z = 3i$ as it is not defined there, however for $z \neq 3i$ we get
	\[
		\frac{z^2 + 9}{z - 3i} = \frac{(z + 3i)(z - 3i)}{z - 3i} = z + 3i,	
	\]
	which is continuous, so
	\[
		\lim_{z \to 3i} \frac{z^2 + 9}{z - 3i} = \lim_{z \to 3i} z + 3i = 6i
	\]
\end{exercise}	
\end{document}
