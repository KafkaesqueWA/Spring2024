\documentclass{article}
\usepackage[margin=2cm]{geometry}
\usepackage[utf8]{inputenc}
\usepackage{amsfonts}
\usepackage{amsmath}
\usepackage{multicol}
\usepackage{amsthm}
\usepackage{amssymb}
\usepackage{graphicx}
\usepackage{tikz}
\usepackage{pgfplots}
\newtheorem{theorem}{Theorum}[section]
\newtheorem{corollary}{Corollary}[theorem]
\newtheorem{lemma}[theorem]{Lemma}
\pgfplotsset{compat=1.18}
\title{Homework Set 1.2}
\author{Warren Atkison}
\date{\today}
\begin{document}

\maketitle

\section*{Exercise 6}
Show that the points $3 + i$, 6, and $4 + 4i$ are the vertices of a right triangle.
\begin{proof}
	If the edges of the triangle formed by these vertices satisfy
	\[
		A^2 + B^2 = C^2
	\]
	Then they form a right a right triangle.
	\begin{multicols}{2}
		
	\begin{tikzpicture}
		\begin{axis}[xmin=0, xmax=7, ymin=0, ymax=5, axis x line=middle, axis y line=middle]
			\addplot[mark=*] coordinates {(3,1)};
			\addplot[mark=*] coordinates {(6,0)};
			\addplot[mark=*] coordinates {(4,4)};
			\addplot[domain=3:6] {-0.333333*x + 2};
			\addplot[domain=3:4] {3*x - 8};
			\addplot[domain=4:6] {-2*x + 12};

	
			
		\end{axis}
	\end{tikzpicture} \\
	Let edge $A$ be the distance from $3 + i$ to $4 + 4i$, $B$ be the distance from $3 + i$ to $6$, and $C$ be the distance from $4 + 4i$ to $6$. Then
	\begin{align*}
		A &= \sqrt{(3 - 4)^2 + (1 - 4)^2} = \sqrt{10} \\
		B &= \sqrt{(3 - 6)^2 + (1 - 0)^2} = \sqrt{10} \\
		C &= \sqrt{(4 - 6)^2 + (4 - 0)^2} = \sqrt{20}     
	\end{align*}
	and
	\[
		\sqrt{10}^2 + \sqrt{10}^2 = \sqrt{20}^2   
	\]
	\end{multicols}
\end{proof}
\section*{Exercise 16}
Prove that if $|z| = 1 (z \neq 1),$ then Re$[1/(1 - z)] = \frac{1}{2}$
\begin{proof}
	Recall that Re$[z] = (z + \overline{z})/2$ and $z\overline{z} = |z|^2$
	\begin{align*}
		\text{Re}[1/(1-z)] = \frac{1}{2}\left(\frac{1}{1-z} + \overline{\left(\frac{1}{1 - z}\right)}\right) &= \frac{1}{2}\left(\frac{1}{1 - z} + \frac{1}{1 - \overline{z}}\right) \\
														     &= \frac{1}{2}\left(\frac{1 - \overline{z} + 1 - z}{1 - \overline{z} - z + z\overline{z}}\right) \\
														     &= \frac{1}{2}\left(\frac{1 - \overline{z} + 1 - z}{1 - \overline{z} - z + |z|^2}\right) \\
														     &= \frac{1}{2}\left(\frac{1 - \overline{z} - z + 1}{1 - \overline{z} - z + 1}\right)\\  &= \frac{1}{2}
	\end{align*}
\end{proof}
\newpage
\section*{Exercise 17}
Let $a_1, a_2, \ldots, a_n$ be real constants. Show that if $z_0$ is a root of the polynomial equation \\ $z^n + a_1z^{n-1} + a_2z^{n-2} + \ldots + a_n = 0$, then so is $\overline{z_0}$.
\begin{proof}
	Let $f(z) = a_0z^n + a_1z^{n-1} + a_2z^{n-2} + \ldots + a_{n-1}z + a_n$. Since $a_1,a_2,\ldots,a_n \in \mathbb{R}$, $a_i = \overline{a_i}$. Then
\begin{align*}
	f(\overline{z_0}) &= a_0\overline{z_0}^n + a_1\overline{z_0}^{n-1} + \ldots + a_n \\
			&= (\overline{a_0})(\overline{z_0})^n + (\overline{a_1})(\overline{z_0})^{n-1} + \ldots + \overline{a_n} \\
			&= \overline{a_0z_0^n} + \overline{a_1z_0}^{n-1} + \ldots + \overline{a_n} \\
			&= \overline{a_0z_0^n + a_1z_0^{n-1} + \ldots + a_n} \\
			&= \overline{0} \\
			&= 0
\end{align*}
\end{proof}
\end{document}
