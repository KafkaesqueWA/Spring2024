\documentclass[12pt]{article}
\linespread{1.2}
\usepackage[margin=2cm]{geometry}
\usepackage[utf8]{inputenc}
\usepackage{amsfonts}
\usepackage{amsmath}
\usepackage{multicol}
\usepackage{amsthm}
\usepackage{amssymb,scrextend}
\usepackage{graphicx,tikz}
\usepackage{esint}
\newtheorem{dfn}{Definition}
\renewcommand{\qed}{\hfill$\blacksquare$}
\let\newproof\proof
\renewenvironment{proof}{\vspace{1em}\begin{addmargin}[2em]{0em}\begin{newproof}}{\end{newproof}\end{addmargin}\qed}
\newenvironment{theorem}[2][Theorem]{\begin{trivlist}
\item[\hskip \labelsep {\bfseries #1} \hskip \labelsep {\bfseries #2.}]}{\end{trivlist}}
\newenvironment{example}[2][Example]{\begin{trivlist}
\item[\hskip \labelsep {\bfseries #1} \hskip \labelsep {\bfseries #2.}]}{\end{trivlist}}
\newenvironment{lemma}[2][Lemma]{\begin{trivlist}
\item[\hskip \labelsep {\bfseries #1} \hskip \labelsep {\bfseries #2.}]}{\end{trivlist}}
\newenvironment{exercise}[2][Exercise]{\begin{trivlist}
\item[\hskip \labelsep {\bfseries #1} \hskip \labelsep {\bfseries #2.}]}{\end{trivlist}}
\newenvironment{problem}[2][Problem]{\begin{trivlist}
\item[\hskip \labelsep {\bfseries #1} \hskip \labelsep {\bfseries #2.}]}{\end{trivlist}}
\newenvironment{corollary}[2][Corollary]{\begin{trivlist}
\item[\hskip \labelsep {\bfseries #1} \hskip \labelsep {\bfseries #2.}]}{\end{trivlist}}
\usepackage{fancyhdr,enumitem,changepage,url}
\pagestyle{fancy}
\author{Warren Atkison}
\date{\today}
\setlength{\headheight}{15pt}
\begin{document}
\fancyhf{}
\fancyhead[L]{Warren Atkison}
\fancyhead[C]{Homework Set 4.4}
\fancyhead[R]{\today}
\fancyfoot[R]{\thepage}

\begin{exercise}{15}
	Evaluate
	\[
		\int_{\Gamma} \frac{z}{(z + 2)(z - 1)}dz
	\]
	Where $\Gamma$ is the circle $|z| = 4$ traversed twice in the clockwise direction.
\end{exercise}
Since $\dfrac{z}{(z + 2)(z - 1)}$ is analytic everywhere except the points $z = -2$ and $z = 1$, we can construct a new contour which encloses these points in small circles, $C_0$ and $C_1$ respectively and splitting apart $\Gamma$ into 4 segments as follows
\begin{center}
\begin{tikzpicture}[>=stealth, scale=.9]
    % Axis
    \draw[->] (-5,0) -- (5,0) node[right] {Re};
    \draw[->] (0,-5) -- (0,5) node[above] {Im};
    
    % Circles
    \draw (0,0) circle (4);
    \draw[->,thick] (0,0) ++(45:4) arc (45:-315:4) node[above right] {$\Gamma_1$};
    \draw[->,thick] (0,0) ++(225:4) arc (225:-135:4) node[below left] {$\Gamma_2$};
    
    \filldraw[fill=black] (-2,0) circle (2pt); %node[below] {$-2$};
    \draw (-2,0) circle (0.5);
    \draw[->,thick] (1,0) ++(45:0.5) arc (45:-315:0.5) node[above right] {$C_1$};
    
    \filldraw[fill=black] (1,0) circle (2pt); %node[below] {$1$};
    \draw (1,0) circle (0.5);
    \draw[->,thick] (-2,0) ++(225:0.5) arc (225:-135:0.5) node[below left] {$C_0$};

    % Points
    \filldraw[fill=black] (-4,0) circle (2pt) node[below left] {$-4$};
    \filldraw[fill=black] (4,0) circle (2pt) node[below right] {$4$};

    % Lines
    \draw[->,thick] (-4,0) -- (-3,0) node[above] {$\Gamma_3$};
    \draw[thick] (-3,0) -- (-2.5,0);
    \draw[thick] (-1.5,0) -- (0.5,0);
    \draw[->,thick] (4,0) -- (2.5,0) node[below] {$\Gamma_4$};
    \draw[thick] (2.5,0) -- (1.5,0);
\end{tikzpicture}
\end{center}
where $\Gamma_1$ is the top half of $\Gamma$ from $-4$ to $4$, $\Gamma_2$ is the bottom half of $\Gamma$ from $4$ to $-4$, $\Gamma_3$ is the path from $-4$ to $4$ creating the top half semi-circles around $-2$ and $1$, and $\Gamma_4$ is the path from $4$ to $-4$ creating bottom half semi-circles around $-2$ and $1$. Then, $\Gamma =2(\Gamma_1 + \Gamma_2)$, and integrating along $\Gamma_1$ is the same as integrating along $\Gamma_3$ and similarly integrating along $\Gamma_2$ is the same as integrating along $\Gamma_4$. Combining the two segments we and taking into account the cancellations along the real axis we find
\begin{align*}
	\int_{\Gamma} \frac{z}{(z+2)(z-1)}dz &= 2\left(\int_{\Gamma_1} + \int_{\Gamma_2}\right)\frac{z}{(z+2)(z-1)}dz \\
					     &= 2\ointclockwise_{C_0} \frac{z}{(z+2)(z-1)}dz + 2\ointclockwise_{C_1} \frac{z}{(z+2)(z -1)} \\
					     &= 2\ointclockwise_{C_0} \left(\frac{2}{3}\cdot\frac{1}{z+2} + \frac{1}{3}\cdot\frac{1}{z-1}\right)dz + 2\ointclockwise_{C_1} \left(\frac{2}{3}\cdot\frac{1}{z+2} + \frac{1}{3}\cdot\frac{1}{z-1}\right)dz \\
					     &= \frac{4}{3}(-2\pi i) +  \frac{2}{3}(-2\pi i) \\
					     &= -4\pi i
\end{align*}
\begin{exercise}{17}
	Evaluate
	\[
		\int_{\Gamma} \frac{2z^2 - z + 1}{(z-1)^2(z+1)}dz
	\]
	where $\Gamma$ is the figure-eight contour traversed once as shown in Fig. 4.49.
\end{exercise}
By the same method from the previous exercise we find that integrating along $\Gamma$ is the same as integrating along $C_0 + C_1$ where $C_0$ is a small circle in the counter-clockwise direction around $z = -1$ and $C_1$ is a small circle in the clockwise direction around $z = 1$ for the function $\dfrac{2z^2 - z + 1}{(z-1)^2(z+1)}$. Note:
\begin{align*}
	\frac{2z^2 - z + 1}{(z-1)^2(z+1)} = \frac{A}{(z-1)^2} + \frac{B}{z-1} + \frac{C}{z+1}
\end{align*}
To find $A$, $B$, and $C$ we can use our method from section 3.1
\begin{align*}
	A &= \lim_{z \to 1} \frac{2z^2 - z + 1}{z+1} = 1 \\
	B &= \lim_{z \to 1} \frac{d}{dz}\left(\frac{2z^2 - z + 1}{z+1}\right) = \lim_{z \to 1} \frac{(z+1)(4z - 1) - (2z^2 - z + 1)}{(z + 1)^2} = \frac{4}{4} = 1 \\
	C &= \lim_{z \to -1} \frac{2z^2 - z + 1}{(z - 1)^2} = \frac{4}{4} = 1
\end{align*}
So
\begin{align*}
	\int_{\Gamma} \frac{2z^2 - z + 1}{(z-1)^2(z+1)}dz &= \ointctrclockwise_{C_0} \frac{2z^2 - z + 1}{(z-1)^2(z+1)}dz + \ointclockwise_{C_1} \frac{2z^2 - z + 1}{(z - 1)^2(z + 1)}dz \\
							  &= \ointctrclockwise_{C_0}\left(\frac{1}{(z - 1)^2} + \frac{1}{z - 1} + \frac{1}{z + 1}\right)dz + \ointclockwise_{C_1}\left(\frac{1}{(z-1)^2} + \frac{1}{z - 1} + \frac{1}{z + 1}\right)dz \\
							  &= 2\pi i - 2\pi i \\
							  &= 0
\end{align*}
\end{document}
