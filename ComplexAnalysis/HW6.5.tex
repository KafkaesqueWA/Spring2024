\documentclass[12pt]{article}
\linespread{1.2}
\usepackage[margin=2cm]{geometry}
\usepackage[utf8]{inputenc}
\usepackage{amsfonts}
\usepackage{amsmath}
\usepackage{multicol}
\usepackage{amsthm}
\usepackage{amssymb,scrextend}
\usepackage{graphicx,tikz}
\newtheorem{dfn}{Definition}
\renewcommand{\qed}{\hfill$\blacksquare$}
\let\newproof\proof
\renewenvironment{proof}{\vspace{1em}\begin{addmargin}[2em]{0em}\begin{newproof}}{\end{newproof}\end{addmargin}\qed}
\newenvironment{theorem}[2][Theorem]{\begin{trivlist}
\item[\hskip \labelsep {\bfseries #1} \hskip \labelsep {\bfseries #2.}]}{\end{trivlist}}
\newenvironment{example}[2][Example]{\begin{trivlist}
\item[\hskip \labelsep {\bfseries #1} \hskip \labelsep {\bfseries #2.}]}{\end{trivlist}}
\newenvironment{lemma}[2][Lemma]{\begin{trivlist}
\item[\hskip \labelsep {\bfseries #1} \hskip \labelsep {\bfseries #2.}]}{\end{trivlist}}
\newenvironment{exercise}[2][Exercise]{\begin{trivlist}
\item[\hskip \labelsep {\bfseries #1} \hskip \labelsep {\bfseries #2.}]}{\end{trivlist}}
\newenvironment{problem}[2][Problem]{\begin{trivlist}
\item[\hskip \labelsep {\bfseries #1} \hskip \labelsep {\bfseries #2.}]}{\end{trivlist}}
\newenvironment{corollary}[2][Corollary]{\begin{trivlist}
\item[\hskip \labelsep {\bfseries #1} \hskip \labelsep {\bfseries #2.}]}{\end{trivlist}}
\usepackage{fancyhdr,enumitem,changepage,url}
\pagestyle{fancy}
\author{Warren Atkison}
\date{\today}
\setlength{\headheight}{15pt}
\begin{document}
\fancyhf{}
\fancyhead[L]{Warren Atkison}
\fancyhead[C]{Homework Set 6.5}
\fancyhead[R]{\today}
\fancyfoot[R]{\thepage}

\begin{exercise}{4}
	Using techniques of residues, verify the following integral formula.
	\[
		\int_{0}^{\infty} \frac{\sin(2x)}{x(x^2 + 1)^2}dx = \pi\left(\frac{1}{2}-\frac{1}{e^2}\right)
	\]
\end{exercise}

Since $\sin(x)$ is even, we have
\[
	\int_{0}^{\infty} \frac{\sin(2x)}{x(x^2 + 1)^2}dx = \frac{1}{2}\int_{-\infty}^{\infty} \frac{\sin(2x)}{x(x^2 + 1)^2}dx
\]

which is the imaginary part of
\begin{align*}
	\int_{-\infty}^{\infty} \frac{e^{2ix}}{x(x^2 + 1)^2}dx.
\end{align*}

Let $f(z) = \dfrac{e^{2iz}}{z(z^2 + 1)^2}$. Since we have a singularity at $z = 0$, we have

\[
	\left(\int_{-\rho}^{r} + \int_{S_r} + \int_{r}^{\rho} + \int_{C_p^{+}}\right)f(z)dz = 2\pi i \text{Res}(f; i)
\]

By Jordan's lemma we have
\[
	\lim_{\rho \to \infty}\int_{C_p^{+}}f(z)dz = 0
\]

and
\[
	\lim_{r \to 0^{+}}\int_{S_r}f(z)dz = -i\pi \text{Res}(f; 0)
\]

so
\[
	\int_{-\infty}^{\infty} f(x)dx = 2\pi i\text{Res}(f; i) + \pi i \text{Res}(f; 0) - 0.
\]

Now we can find the residues. $\text{Res}(f; 0) = \lim_{z \to 0} f(z) = 1$, and
\begin{align*}
	\text{Res}(f; i) &= \lim_{z \to i} \frac{d}{dz}\left(\frac{e^{2iz}}{z(z + i)^2}\right) \\
			 &= \frac{-4i(2ie^{-2}) - e^{-2}(-8)}{16} \\
			 &= \frac{-1}{e^2}.
\end{align*}

Thus
\begin{align*}
	\int_{0}^{\infty} \frac{\sin(2x)}{x(x^2+1)^2}dx &= \frac{1}{2}\text{Im}\left(\frac{-2\pi i}{e^2} + \pi i\right) \\
							&= \pi\left(\frac{1}{2} - \frac{1}{e^2}\right)
\end{align*}

\begin{exercise}{5}
	Using techniques of residues, verify the following integral formula.
	\[
		\int_{0}^{\infty} \frac{\cos(x) - 1}{x^2}dx = -\frac{\pi}{2}
	\]
\end{exercise}	

since $\cos(x) - 1$ is and even function we have
\[
	\int_{0}^{\infty} \frac{\cos(x) - 1}{x^2}dx = \frac{1}{2}\int_{-\infty}^{\infty} \frac{\cos(x) - 1}{x^2}dx
\]

which is the real part of
\[
	\int_{-\infty}^{\infty} \frac{e^{ix} - 1}{x^2}dx.
\]

Let $f(z) = \dfrac{e^{iz} - 1}{z^2}$. Since we have a singularity at $z = 0$ we have
\[
	\left(\int_{-\rho}^{r} + \int_{S_r} + \int_{r}^{\rho} + \int_{C_p^{+}}\right)f(z)dz = 0
\]

By Jordan's lemma we have
\[
	\lim_{\rho \to \infty}\int_{C_p^{+}}f(z)dz = 0
\]

and
\[
	\lim_{r \to 0^{+}}\int_{S_r}f(z)dz = -i\pi \text{Res}(f; 0)
\]

so
\[
	\int_{-\infty}^{\infty} f(x)dx = \pi i \text{Res}(f; 0) - 0.
\]

Now we can find the residues.
\begin{align*}
	\text{Res}(f; 0) &= \lim_{z \to 0} \frac{d}{dz}(e^{iz} - 1) = i
\end{align*}

Thus
\begin{align*}
	\int_{0}^{\infty} \frac{\cos(x) - 1}{x^2} &= \frac{1}{2}\text{Re}\left(\pi i (i)\right) = -\frac{\pi}{2}
\end{align*}
\end{document}
