\documentclass[12pt]{article}
\linespread{1.2}
\usepackage[margin=2cm]{geometry}
\usepackage[utf8]{inputenc}
\usepackage{amsfonts}
\usepackage{amsmath}
\usepackage{multicol}
\usepackage{amsthm}
\usepackage{amssymb,scrextend}
\usepackage{graphicx,tikz}
\newtheorem{dfn}{Definition}
\renewcommand{\qed}{\hfill$\blacksquare$}
\let\newproof\proof
\renewenvironment{proof}{\vspace{1em}\begin{addmargin}[2em]{0em}\begin{newproof}}{\end{newproof}\end{addmargin}\qed}
\newenvironment{theorem}[2][Theorem]{\begin{trivlist}
\item[\hskip \labelsep {\bfseries #1} \hskip \labelsep {\bfseries #2.}]}{\end{trivlist}}
\newenvironment{example}[2][Example]{\begin{trivlist}
\item[\hskip \labelsep {\bfseries #1} \hskip \labelsep {\bfseries #2.}]}{\end{trivlist}}
\newenvironment{lemma}[2][Lemma]{\begin{trivlist}
\item[\hskip \labelsep {\bfseries #1} \hskip \labelsep {\bfseries #2.}]}{\end{trivlist}}
\newenvironment{exercise}[2][Exercise]{\begin{trivlist}
\item[\hskip \labelsep {\bfseries #1} \hskip \labelsep {\bfseries #2.}]}{\end{trivlist}}
\newenvironment{problem}[2][Problem]{\begin{trivlist}
\item[\hskip \labelsep {\bfseries #1} \hskip \labelsep {\bfseries #2.}]}{\end{trivlist}}
\newenvironment{corollary}[2][Corollary]{\begin{trivlist}
\item[\hskip \labelsep {\bfseries #1} \hskip \labelsep {\bfseries #2.}]}{\end{trivlist}}
\usepackage{fancyhdr,enumitem,changepage,url}
\pagestyle{fancy}
\author{Warren Atkison}
\date{\today}
\setlength{\headheight}{15pt}
\begin{document}
\fancyhf{}
\fancyhead[L]{Warren Atkison}
\fancyhead[C]{Homework Set 4.1}
\fancyhead[R]{\today}
\fancyfoot[R]{\thepage}

\begin{exercise}{4}
	Show that the range of the function $z(t) = t^3 + it^6$, $-1 \le t \le 1$, is a smooth curve even though the given parametrization is not admissible.
\end{exercise}
\begin{proof}
	Let $u = t^3$, then
	\[
		z(u) = u + iu^2, \text{ for } (-1)^3 \le u \le (1)^3
	\]
	Then
	\begin{align*}
		\frac{dz(u)}{du} = 1 + 2iu \neq 0
	\end{align*}
	for all $-1 \le u \le 1$. Therefore the range of $z(u)$ is a smooth curve and thus the range of $z(t)$ is also a smooth curve. 
\end{proof}
\begin{exercise}{8}
	Parametrize the contour $\Gamma$ indicated in in Fig. 4.14. Also give a parametrization for the opposite contour $-\Gamma$
\end{exercise}	
We have 2 sections to consider, $\gamma_1$ and $\gamma_2$. $\gamma_1$ goes from $(-2,2i)$ to $(-1,0)$, which can be describe by
\[
	z_1(t) = t - 2 + 2i(1 - t) \text{ for } 0 \le t \le 1.
\]
$\gamma_2$ is a semicircle with radius 1 about the origin going clockwise, which can be described by
\[
	z_2(t) = -\cos(t) + i\sin(t) \text{ for } 0 \le t \le \pi.
\]
We can shift the bounds on $t$ for $\gamma_2$, and
\[
	z(t) = \begin{cases}
		t-2 + 2i(1 - t) & 0 \le t \le 1 \\
		-\cos(t-1) + i\sin(t-1) & 1 \le t \le 1 + \pi 
	\end{cases}
\]
describes $\Gamma$. For $-\Gamma$, we reverse the direction, so
\[
	z(t) = \begin{cases}
		\cos(t) + i\sin(t) & 0 \le t \le \pi \\
		-(t - \pi) - 1 + 2i(t - \pi) & \pi \le t \le \pi + 1
	\end{cases}
\]
describes $-\Gamma$.
\begin{exercise}{10}
	Using an admissible parametrization verify from formula (1) that
	\begin{itemize}
		\item[(a)] the length of the line segment from $z_1$ to $z_2$ is $|z_2 - z_1|$;
			\begin{proof}
				Let $\gamma$ be our contour. We can describe $\gamma$ as follows:
				\begin{align*}
					z(t) &= z_1 + t(z_2 - z_1) = x_1 + iy_1 + t((x_2 + iy_2) - (x_1 + iy_1)) \\
					     &= t(x_2 - x_1) + x_1 + i(t(y_2 - y_1) + y_1).
				\end{align*}
				for $0 \le t \le t$. Let $s(t)$ be the lenght of $\gamma$, then
				\[
					\frac{ds}{dt} = \sqrt{(x_2 - x_1)^2 + (y_2 - y_1)^2} = |z_2 - z_1|
				\]
				and
				\begin{align*}
					s = \int_{0}^{1} |z_2 - z_1| dt = |z_2 - z_1|
				\end{align*}
			\end{proof}
		\item[(b)] the length of the circle $|z - z_0| = r$ is $2\pi r$
			\begin{proof}
				Let $\gamma$ be our contour. We can describe $\gamma$ as follows:
				\begin{align*}
					z(t) = r\cos(t) + x_0 + i(r\sin(t) + y_0)
				\end{align*}
				for $0 \le t \le 2\pi$. Let $s(t)$ be the length of $\gamma$, then
				\begin{align*}
					\frac{ds}{dt} &= \sqrt{(-r\sin(t))^2 + (r\cos(t))^2} = \sqrt{r^2(\sin^2(t) + \cos^2(t))} = r.
				\end{align*}
				so
				\begin{align*}
					s = \int_{0}^{2\pi}rdt = 2\pi r
				\end{align*}
			\end{proof}
	\end{itemize}
\end{exercise}
\end{document}
