\documentclass[12pt]{article}
\linespread{1.2}
\usepackage[margin=2cm]{geometry}
\usepackage[utf8]{inputenc}
\usepackage{amsfonts}
\usepackage{amsmath}
\usepackage{multicol}
\usepackage{amsthm}
\usepackage{amssymb,scrextend}
\usepackage{graphicx,tikz}
\newtheorem{dfn}{Definition}
\renewcommand{\qed}{\hfill$\blacksquare$}
\let\newproof\proof
\renewenvironment{proof}{\vspace{1em}\begin{addmargin}[2em]{0em}\begin{newproof}}{\end{newproof}\end{addmargin}\qed}
\newenvironment{theorem}[2][Theorem]{\begin{trivlist}
\item[\hskip \labelsep {\bfseries #1} \hskip \labelsep {\bfseries #2.}]}{\end{trivlist}}
\newenvironment{example}[2][Example]{\begin{trivlist}
\item[\hskip \labelsep {\bfseries #1} \hskip \labelsep {\bfseries #2.}]}{\end{trivlist}}
\newenvironment{lemma}[2][Lemma]{\begin{trivlist}
\item[\hskip \labelsep {\bfseries #1} \hskip \labelsep {\bfseries #2.}]}{\end{trivlist}}
\newenvironment{exercise}[2][Exercise]{\begin{trivlist}
\item[\hskip \labelsep {\bfseries #1} \hskip \labelsep {\bfseries #2.}]}{\end{trivlist}}
\newenvironment{problem}[2][Problem]{\begin{trivlist}
\item[\hskip \labelsep {\bfseries #1} \hskip \labelsep {\bfseries #2.}]}{\end{trivlist}}
\newenvironment{corollary}[2][Corollary]{\begin{trivlist}
\item[\hskip \labelsep {\bfseries #1} \hskip \labelsep {\bfseries #2.}]}{\end{trivlist}}
\usepackage{fancyhdr,enumitem,changepage,url}
\pagestyle{fancy}
\author{Warren Atkison}
\date{\today}
\setlength{\headheight}{15pt}
\begin{document}
\fancyhf{}
\fancyhead[L]{Warren Atkison}
\fancyhead[C]{Homework Set 4.3}
\fancyhead[R]{\today}
\fancyfoot[R]{\thepage}

\begin{exercise}{1}
	Calculate each of the following integrals along the inducated contours. Observe that a standard table of integrals can be used. Explain why.
\end{exercise}
Because the following functions are continuous in $\mathbb{C}$ have antiderivatives in $\mathbb{C}$, then for any contour in $\mathbb{C}$
\[
	\int_{\Gamma}f(z)dz = F(z_T) - F(z_1)
\]
where $z_1$ is the initail point of a contour and $z_T$ is the terminal point of a contour.
\begin{itemize}
	\item[(a)] $\int_{\Gamma} (3z^2 - 5z + i)dz$ along the line segment from $z = i$ to $z = 1$.
		\begin{align*}
			\int_{\Gamma} (3z^2 - 5z + i) &= \left[z^3 - \frac{5}{2}z^2 + iz\right]_{i}^{1} \\
						      &= \left(1 - \frac{5}{2} + i\right) - \left(-i + \frac{5}{2} - 1\right) \\
						      &= -3 + 2i 
		\end{align*}
	\item[(e)] $\int_{\Gamma} \sin^2 z\cos(z)dz$ along the contour in Fig. 4.24. \\
		Let $u = \sin(z)$, then $du = \cos(z)dz$.
		\begin{align*}
			\int_{\Gamma} \sin^2(z)\cos(z)dz &= \int_{\Gamma} u^2du \\
							 &= \left[\frac{u^3}{3}\right]_{\sin(\pi)}^{\sin(i)} \\
							 &= \frac{\sin^3(i)}{3} \\
							 &= -\frac{i}{3}\sinh^3(1)
		\end{align*}
\end{itemize}
\begin{exercise}{7}
	Show that if $C$ is a positevely oriented circle and $z_0$ lies outside $C$, then
	\[
		\int_{C}\frac{dz}{z - z_0} = 0
	\]
\end{exercise}	
\begin{proof}
	Let $s$ be the line from $z_0$ passing through center of $C$ and extending onward. Since $z_0$ lies outside of $C$, this line always exists and has a direction. Now let $-s$ be the line from $z_0$ in the opposite direction of $s$ extending forever. Then $D = \mathbb{C} - (-s)$ forms a domain in which $\dfrac{1}{z - z_0}$ is continuous and has an antiderivative, namely $\log(z - z_0)$ with branch cut $-s$. Also, clearly $C \in D$, so by Theorem 7 the integral of $C$.
\end{proof}
\begin{center}
	\begin{tikzpicture}

% Axis
\draw[->] (-4,0) -- (4,0) node[right] {$\text{Re}$};
\draw[->] (0,-3) -- (0,4) node[above] {$\text{Im}$};

% Circle
\draw[blue] (-3,2) circle (1);

% Center
\filldraw[black] (-3,2) circle (2pt) node[anchor=south east] {$C$};
\filldraw[black] (2,-1) circle (2pt) node[anchor=south east] {$z_0$};
\draw[red, ->] (2,-1) -- (5.333,-3);

\end{tikzpicture}
\end{center}
\begin{exercise}{12}
	Let $f$ be an analytic function with a continuous derivative satisfying $|f'(z)| \le M$ for all $z$ in the disk $D: |z| <1$. Show that
	\[
		|f(z_2) - f(z_1)| \le M|z_2 - z_1| \quad (z_1,z_2 \in D).
	\]
\end{exercise}	
\begin{proof}
	Let $\Gamma$ be the sine segment from $z_1$ to $z_2$, then
	\begin{align*}
		|f(z_2) - f(z_1)| &= \left|\int_{\Gamma} f'(z) dz \right| \\
				  &\le \int_{\Gamma} |f'(z)| dz \\
				  &\le  \int_{\Gamma} M dz \\
				  &= M(z_2 - z_1) \\
				  &\le M|z_2 - z_1|
	\end{align*}
\end{proof}
\end{document}
