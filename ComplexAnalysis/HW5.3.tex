\documentclass[12pt]{article}
\linespread{1.2}
\usepackage[margin=2cm]{geometry}
\usepackage[utf8]{inputenc}
\usepackage{amsfonts}
\usepackage{amsmath}
\usepackage{multicol}
\usepackage{amsthm}
\usepackage{amssymb,scrextend}
\usepackage{graphicx,tikz}
\newtheorem{dfn}{Definition}
\renewcommand{\qed}{\hfill$\blacksquare$}
\let\newproof\proof
\renewenvironment{proof}{\vspace{1em}\begin{addmargin}[2em]{0em}\begin{newproof}}{\end{newproof}\end{addmargin}\qed}
\newenvironment{theorem}[2][Theorem]{\begin{trivlist}
\item[\hskip \labelsep {\bfseries #1} \hskip \labelsep {\bfseries #2.}]}{\end{trivlist}}
\newenvironment{example}[2][Example]{\begin{trivlist}
\item[\hskip \labelsep {\bfseries #1} \hskip \labelsep {\bfseries #2.}]}{\end{trivlist}}
\newenvironment{lemma}[2][Lemma]{\begin{trivlist}
\item[\hskip \labelsep {\bfseries #1} \hskip \labelsep {\bfseries #2.}]}{\end{trivlist}}
\newenvironment{exercise}[2][Exercise]{\begin{trivlist}
\item[\hskip \labelsep {\bfseries #1} \hskip \labelsep {\bfseries #2.}]}{\end{trivlist}}
\newenvironment{problem}[2][Problem]{\begin{trivlist}
\item[\hskip \labelsep {\bfseries #1} \hskip \labelsep {\bfseries #2.}]}{\end{trivlist}}
\newenvironment{corollary}[2][Corollary]{\begin{trivlist}
\item[\hskip \labelsep {\bfseries #1} \hskip \labelsep {\bfseries #2.}]}{\end{trivlist}}
\usepackage{fancyhdr,enumitem,changepage,url}
\pagestyle{fancy}
\author{Warren Atkison}
\date{\today}
\setlength{\headheight}{15pt}
\begin{document}
\fancyhf{}
\fancyhead[L]{Warren Atkison}
\fancyhead[C]{Homework Set 5.3}
\fancyhead[R]{\today}
\fancyfoot[R]{\thepage}

\begin{exercise}{4}
	Does there exist a power series $\sum_{j=0}^{\infty} a_jz^j$ that converges at $z = 2 + 3i$ and diverges at $z = 3-i$?
\end{exercise}

No. From Theorem 10 we know that if a power series converges at a point then it must converge everyhwere inside the circle where the point lies. Since $3 - i$ is inside the circle $|z| \le |2 + 3i|$, then if a power series converges at $2 + 3i$ it must converge at $3 - i$.

\begin{exercise}{13a}
	The following initial-value problem has a unique solution that is analytic at the origin. Find the power series expansion $\sum_{j=0}^{\infty} a_jz^j$ of the solution by determining a recurrence relation for the coefficients $a_j$.
	\[
		\begin{cases}
			\dfrac{d^2f}{dz^2} - z\dfrac{df}{dz} - f = 0 \\
			f(0) = 1, \quad f'(0) = 0
		\end{cases}
	\]
\end{exercise}
First we have
\begin{align*}
	f''(0) &= zf'(0) + f(0) = 1 \\
	f'''(0) &= zf''(0) + 2f'(0) = 0 \\
	f^{(4)}(0) &= zf^{(3)}(0) + 3f^{(2)}(0) = 3 \\
	f^{(5)}(0) &= zf^{(4)}(0) + 4f^{(3)}(0) = 0 \\
	f^{(6)}(0) &= zf^{(5)}(0) + 5f^{(4)}(0) = 15 \\
		   &~ ~\vdots
\end{align*}
and in general
\[
	f^{(j)}(0) = (j-1)f^{(j-2)}(0) = \begin{cases}
		1 \cdot 3 \cdot 5 \cdot \ldots \cdot (j-1) & \text{when $j$ is even} \\
		0 & \text{when $j$ is odd}
	\end{cases}.
\]
Note that:
\[
	\prod_{k=0}^j (2k - 1) = \frac{(2j)!}{2^jj!}.
\]
Thus, using the Maclaurin series we have
\[
	f(z) = \sum_{j=0}^{\infty} \frac{(2j)!}{2^jj!}\cdot\frac{z^{2j}}{(2j)!} = \sum_{j=0}^{\infty} \frac{(z^2/2)^j}{j!} = e^{z^2/2}
\]
\end{document}
