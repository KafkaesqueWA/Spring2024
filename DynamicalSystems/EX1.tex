\documentclass[12pt]{article}
\linespread{1.2}
\usepackage[margin=2cm]{geometry}
\usepackage[utf8]{inputenc}
\usepackage{amsfonts}
\usepackage{amsmath}
\usepackage{multicol}
\usepackage{amsthm}
\usepackage{amssymb,scrextend}
\usepackage{graphicx,tikz,pgfplots}
\pgfplotsset{compat=1.18}
\usetikzlibrary{arrows}
\newtheorem{dfn}{Definition}
\renewcommand{\qed}{\hfill$\blacksquare$}
\let\newproof\proof
\renewenvironment{proof}{\vspace{1em}\begin{addmargin}[2em]{0em}\begin{newproof}}{\end{newproof}\end{addmargin}\qed}
\newenvironment{theorem}[2][Theorem]{\begin{trivlist}
\item[\hskip \labelsep {\bfseries #1} \hskip \labelsep {\bfseries #2.}]}{\end{trivlist}}
\newenvironment{example}[2][Example]{\begin{trivlist}
\item[\hskip \labelsep {\bfseries #1} \hskip \labelsep {\bfseries #2.}]}{\end{trivlist}}
\newenvironment{lemma}[2][Lemma]{\begin{trivlist}
\item[\hskip \labelsep {\bfseries #1} \hskip \labelsep {\bfseries #2.}]}{\end{trivlist}}
\newenvironment{exercise}[2][Exercise]{\begin{trivlist}
\item[\hskip \labelsep {\bfseries #1} \hskip \labelsep {\bfseries #2.}]}{\end{trivlist}}
\newenvironment{problem}[2][Problem]{\begin{trivlist}
\item[\hskip \labelsep {\bfseries #1} \hskip \labelsep {\bfseries #2.}]}{\end{trivlist}}
\newenvironment{corollary}[2][Corollary]{\begin{trivlist}
\item[\hskip \labelsep {\bfseries #1} \hskip \labelsep {\bfseries #2.}]}{\end{trivlist}}
\usepackage{fancyhdr,enumitem,changepage,url}
\pagestyle{fancy}
\author{Warren Atkison}
\date{\today}
\setlength{\headheight}{15pt}
\begin{document}
\fancyhf{}
\fancyhead[L]{Warren Atkison}
\fancyhead[C]{Exam 1}
\fancyhead[R]{\today}
\fancyfoot[R]{\thepage}

\begin{problem}{1}

\end{problem}
\begin{itemize}
	\item[(a)] The set of $\{f^{i}(x) ~|~ i \ge 0\}$
	\item[(b)] The orbit \textit{begins} to repeat itself after $n$ iterations, that is $x_n = x_0$ and $x_n \neq x_i$ for all $0 \le i \le n - 1$
	\item[(c)] $|f'(x_0) | < 1$ 
	\item[(d)] Every point in $B$ is either dense in $A$ or a limit of $A$.
\end{itemize}	
\begin{problem}{2}
	
\end{problem}
\begin{itemize}
	\item[(a)]
		\[
			11111\ldots - 001111\ldots = 110000\ldots
		\]
	\item[(b)] We took every string of length $n$, and appended it to our point in $\Sigma$.
		\[
			0~1~00~01~10~11~000~001~010~011~100~101~110~111~0000\ldots	
		\]
\end{itemize}
\begin{problem}{3}
	
\end{problem}
\begin{itemize}
	\item[(a)] True
		\begin{align*}
			s &= 0.0222\ldots \\
			3s &= 0.2222\ldots \\
			2s &= 0.2 \\
			s &= 0.1
		\end{align*}
	\item[(b)] True, for example the rationals in $[0,1]$ have length 0 and are dense in the real interval $[0,1]$. 
	\item[(c)] False, $1/9 \le 1/6 \le 2/9$.
	\item[(d)] False, $p_{+}$ and $p_{-}$ are fixed.
\end{itemize}
\begin{problem}{4}

\end{problem}
\begin{itemize}
	\item[(a)]
		\begin{itemize}
			\item Periodic points are dense in $\Lambda$
			\item All orbits are transitive
			\item The system is sensitive to initial conditions (implied by the first 2).
		\end{itemize}
	\item[(b)]
		\begin{itemize}
			\item There is only one periodic point at $x = 0$, which is not dense in $\lambda$.
			\item $x = 0$ is an attracting fixed point, so orbits aren't transitive. For example, the orbit of $x_0 = 1$ will not visit the neighborhood of anything greater than 1.
			\item Every point goes to 0, so the system is not sensitive under initial conditions.
		\end{itemize}
\end{itemize}
\begin{problem}{5}
	
\end{problem}
\begin{align*}
	x = \frac{1}{2}(x^3 + x) \implies 2x = x^3 + x \implies x^3 - x = 0 \implies x(x^2 - 1) = 0
\end{align*}
So $x = 0,1,-1$ are our fixed points.
\[
	f'(x) = \frac{3}{2}x^2 + \frac{1}{2}
\]
$|f'(0)| = 1/2$ so $x = 0$ is attracting, $|f'(1)| = 2$ so $x = 1$ is repelling, and $|f'(-1)| = 2$ so $x = -1$ is also repelling.
\begin{problem}{6}
	
\end{problem}
$f(0) = 1,~f(1) = 2,~= f(2) = 0$, so the orbit of 0 is a period 3 orbit.
\[
	f'(x) = -3x + 5/2
\]
\begin{align*}
	|f^3(0)| = |f'(0)\cdot f'(1)\cdot f'(2)| = |(5/2)(-1/2)(-7/2)| = |35/2| > 1
\end{align*}
so the period 3 orbit of $x = 0$ is repelling.
\begin{problem}{7}
	
\end{problem}
\[
	h \circ f = g \circ h
\]
Let $h(x) = cx$, then
\begin{align*}
	h(f(x)) &= cx^3 & g(h(x)) = \frac{c^3}{4}x^3
\end{align*}
\begin{align*}
	4c = c^3 \implies c^3 - 4c = 0 \implies c(c^2 - 4) = 0
\end{align*}
so $c = 0,2,-2$. Let $c = 2$. Then
\[
	h \circ f = 2x^3 = \frac{2^3}{4}x^3 = g \circ h
\]
\begin{problem}{8}
	This is a saddle node bifurcation, since a fixed point pops into existence and then there are 2 fixed points from that point on. A graph of the diagram looks as follows.
	\begin{center}
		\begin{tikzpicture}[scale=1.5]
  % Axes
  \draw[->] (-2,0) -- (2,0) node[right] {$\lambda$};
  \draw[->] (0,-2) -- (0,2) node[above] {$x$};
  
  % Parabola
  \draw[domain=-1.5:1.5,smooth,variable=\y,blue] plot ({- \y*\y}, {\y});
  
  % Label
\end{tikzpicture}
	\end{center}
\end{problem}
\begin{problem}{9}
	There is a fixed point at $x = 0$, and 2 period 2 points at $x = \pm1$. When $x_0 < -1$, the orbit diverges to $-\infty$. When $-1 < x_0 < 1$ and $x_0 \neq 0$, the orbit converges to 0. When $x > 1$, the orbit diverges to $\infty$. 
\begin{center}
				\begin{tikzpicture}[scale = 2]
				\draw[latex-] (-2.5,0) -- (2.5,0) ;
				\draw[-latex] (-2.5,0) -- (2.5,0) ;
				\foreach \x in  {-2,-1,0,1,2}
				\draw[shift={(\x,0)},color=black] (0pt,3pt) -- (0pt,-3pt);
				\foreach \x in {-2,-1,0,1,2}
				\draw[shift={(\x,0)},color=black] (0pt,0pt) -- (0pt,-3pt) node[below] 
				{$\x$};
				\draw[*-] (-0.04,0) -- (1,0);
				\draw[*-] (-1.04,0) -- (1,0);
				\draw[*-] (1.04,0) -- (-1,0);
				\draw[latex-] (0.5,0) -- (1,0);
				\draw[latex-] (1.75,0) -- (-1,0);
				\draw[latex-] (-0.5,0) -- (-2,0);
				\draw[latex-] (-1.75,0) -- (1,0);

				
			\end{tikzpicture}
		\end{center}
\end{problem}
\end{document}
