
\documentclass[12pt]{article}
\linespread{1.2}
\usepackage[margin=2cm]{geometry}
\usepackage[utf8]{inputenc}
\usepackage{amsfonts}
\usepackage{amsmath}
\usepackage{multicol}
\usepackage{amsthm}
\usepackage{amssymb,scrextend}
\usepackage{graphicx,tikz}
\newtheorem{dfn}{Definition}
\renewcommand{\qed}{\hfill$\blacksquare$}
\let\newproof\proof
\renewenvironment{proof}{\vspace{1em}\begin{addmargin}[2em]{0em}\begin{newproof}}{\end{newproof}\end{addmargin}\qed}
\newenvironment{theorem}[2][Theorem]{\begin{trivlist}
\item[\hskip \labelsep {\bfseries #1} \hskip \labelsep {\bfseries #2.}]}{\end{trivlist}}
\newenvironment{example}[2][Example]{\begin{trivlist}
\item[\hskip \labelsep {\bfseries #1} \hskip \labelsep {\bfseries #2.}]}{\end{trivlist}}
\newenvironment{lemma}[2][Lemma]{\begin{trivlist}
\item[\hskip \labelsep {\bfseries #1} \hskip \labelsep {\bfseries #2.}]}{\end{trivlist}}
\newenvironment{exercise}[2][Exercise]{\begin{trivlist}
\item[\hskip \labelsep {\bfseries #1} \hskip \labelsep {\bfseries #2.}]}{\end{trivlist}}
\newenvironment{problem}[2][Problem]{\begin{trivlist}
\item[\hskip \labelsep {\bfseries #1} \hskip \labelsep {\bfseries #2.}]}{\end{trivlist}}
\newenvironment{corollary}[2][Corollary]{\begin{trivlist}
\item[\hskip \labelsep {\bfseries #1} \hskip \labelsep {\bfseries #2.}]}{\end{trivlist}}
\usepackage{fancyhdr,enumitem,changepage,url}
\pagestyle{fancy}
\author{Warren Atkison}
\date{\today}
\setlength{\headheight}{15pt}
\begin{document}
\fancyhf{}
\fancyhead[L]{15 Puzzle}
\fancyhead[C]{Problems}
\fancyhead[R]{\today}
\fancyfoot[R]{\thepage}

\begin{problem}{1}
	Scramble your puzzle and solve it.
\end{problem}
\vspace{50pt}
\begin{problem}{2}
	Is it possible to reverse the order of the puzzle?
\end{problem}
\vspace{50pt}
\begin{problem}{3}
	Is every scramble solveable? If we took out each peice and randomly inserted them into the 4x4 grid, could we get back to the starting point every time?
\end{problem}
\vspace{50pt}
\begin{problem}{4}
	Suppose two configurations are connected if you can get from one to another with a sequence of legal moves. How many different "compnents" are there, or configurations that cannot be connected?
\end{problem}
\vspace{50pt}
\begin{problem}{5}
	Develop an ad-hoc algorithm to solve the puzzle.
\end{problem}
\vspace{50pt}
\begin{problem}{6}
	If you add an edge from any two vertices, when does it allow you to complete a non-solveable puzzle?
\end{problem}
\vspace{50pt}
\end{document}
